\section{Research Method}
\subsubsection{Identify Research Objectives}
The primary objective of our research is to evaluate the extent to which pair programming has contributed to the success of our project. Pair programming, a collaborative software development technique where two programmers work together at one workstation, has been implemented within our team as a means to improve code quality, increase productivity, and enhance overall project outcomes. Our research aims to provide evidence-based insights into the efficacy of pair programming in achieving these objectives.
\subsubsection{Research Questions}
\begin{enumerate}
    \item To what extent has pair programming influenced the efficiency and productivity of our development process?
    \item How has pair programming affected the quality and correctness of the code produced?
    \item What are the perceived benefits and challenges of pair programming as reported by team members?
\end{enumerate}

\subsubsection{Research Methodology: Surveys}
To address these research questions, we have chosen to utilize surveys as our research methodology. Surveys offer a structured approach to collecting feedback and insights from team members engaged in pair programming. By administering surveys, we can gather quantitative and qualitative data to assess the impact of pair programming on various aspects of our project.
 
\subsubsection{Data Analysis and Interpretation}
Once survey responses have been collected, we will conduct a comprehensive analysis to identify trends, patterns, and correlations in the data. Quantitative data will be analyzed using statistical techniques to measure the impact of pair programming on project outcomes. Qualitative responses will be coded and thematically analyzed to extract key themes and insights.

\subsection{Expected Outcomes}
Through this research, we anticipate gaining a deeper understanding of how pair programming has influenced our project. We expect to uncover insights into the benefits and challenges associated with pair programming, as well as its overall impact on productivity, code quality, and team dynamics. These findings will inform decision-making and provide valuable guidance for optimizing our development process in future projects.

\subsection{Evaluation Criteria}

In evaluating our practice of pair programming, we utilize the following criteria:

\subsubsection{Efficiency} 
This criterion assesses how effectively pair programming enhances productivity and reduces development time. Efficiency encompasses factors such as code quality, task completion rates, and overall team productivity. By monitoring efficiency, we aim to identify areas for optimization and improvement in our development process.

\subsubsection{Correctness of Code} 
Ensuring the correctness of the code is paramount in software development. This criterion evaluates the accuracy and reliability of the code produced through pair programming. It considers factors such as adherence to coding standards, absence of bugs and errors, and the effectiveness of testing procedures. By prioritizing correctness, we aim to deliver high-quality software that meets the needs and expectations of our stakeholders.
