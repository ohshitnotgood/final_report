\section{Project Results}
We conducted a survey among members of the project group. Participating in our survey were Zaid, Kristians, Praanto and Rushil. Due to insufficient participant numbers, the report authors also took part in the survey. This group was chosen because our involvement in the project provides a deeper understanding of the survey outcomes. We received the following responses to our Survey Questions:
\subsection{To what extent has pair programming influenced the efficiency and productivity of our development process?}
\begin{itemize}
    \item Praanto: Pair programming has significantly improved our performance and has been instrumental in leveling the skill gap between several members of the team. While it may sometimes seem cumbersome, the collaborative nature of pair programming fosters knowledge sharing and faster problem-solving, ultimately enhancing our overall efficiency.
    \item Kristians: Despite the perception that pair programming halves the number of tasks that can be done simultaneously, I've observed that the time to complete a task is usually lessened. This is because more otherwise-missable issues are caught during the development process, rather than during testing. The collaborative nature of pair programming enables us to catch and address potential issues in real-time, leading to more efficient development cycles.
    \item Rushil: Although pair programming may initially slow down more experienced developers, it allows for more tasks to be parallelized as we progress. This iterative improvement in task parallelization ultimately enhances our overall efficiency. Additionally, pair programming facilitates continuous learning and skill development among team members, further contributing to our productivity in the long run.
    \item Zaid: Personally, I prefer not working simultaneously in real-time during pair programming sessions. Instead, I believe code reviews are more effective, as they allow for a more thorough examination of the code without the potential distractions or conflicts that may arise during real-time collaboration.
\end{itemize}
\subsection{How has pair programming affected the quality and correctness of the code produced?}
\begin{itemize}
    \item Praanto:  Pair programming has indeed reduced our efficiency, but it has significantly increased the accuracy and reliability of our code. There is a particular phrase I often use, which holds significant meaning for me. However, quoting myself directly would be rather inelegant. By having two sets of eyes on the code as it's being written, we catch and address potential issues earlier in the development process, resulting in code with fewer critical bugs.
    \item Kristians: Overall, the practice of pair programming has led to code with fewer critical bugs. Having someone else look over your code in real-time helps catch issues that might have otherwise been missed, ultimately resulting in higher-quality code. This collaborative approach to coding ensures that potential errors are identified and resolved promptly, contributing to the overall correctness of our codebase.
    \item Rushil: Pair programming plays a crucial role in improving the quality of our code. By working collaboratively, any mistakes or inconsistencies are more likely to be caught at an earlier stage, reducing the likelihood of introducing bugs into the codebase. This proactive approach to code review and validation enhances the overall correctness and reliability of our code.
    \item Zaid: As I did not participate in pair programming sessions, I do not have a particular opinion on how it has affected the quality and correctness of the code produced. However, I acknowledge that the collaborative nature of pair programming can offer valuable insights and opportunities for code improvement.
\end{itemize}
\subsection{What are the perceived benefits and challenges of pair programming?}
\begin{itemize}
    \item Praanto: While pair programming facilitates quicker error resolution, it can sometimes be cumbersome to coordinate and collaborate effectively. However, the benefits of pair programming, such as knowledge sharing and skill leveling, outweigh these challenges in the long run.
    \item Kristians: The benefits of pair programming include increased accuracy and efficiency, particularly when the pair consists of compatible personalities. However, challenges may arise if the pair consists of incompatible personalities, which can hinder efficiency. Despite these challenges, the overall benefits of pair programming in terms of code quality and collaboration make it a valuable practice for our team.
    \item Rushil: Although pair programming may lead to a decrease in efficiency, especially if the programmers have significant differences in skill sets, it offers numerous benefits such as improved code quality and error detection. The collaborative nature of pair programming fosters teamwork and knowledge sharing, ultimately contributing to the success of our projects.
    \item Zaid: As I did not engage in pair programming, I do not have a particular opinion on the perceived benefits and challenges of this practice. However, I recognize that pair programming can offer advantages in terms of code quality and collaboration, while also presenting challenges related to coordination and compatibility among team members.
\end{itemize}